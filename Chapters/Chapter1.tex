% Chapter Template

\chapter{Introducere/Introduction/Einleitung} % Main chapter title

\label{Chapter1} % Change X to a consecutive number; for referencing this chapter elsewhere, use \ref{ChapterX}

\lhead{Chapter 1. \emph{Introducere/Introduction/Einleitung}} % Change X to a consecutive number; this is for the header on each page - perhaps a shortened title

%----------------------------------------------------------------------------------------
%	RO
%----------------------------------------------------------------------------------------
\section{Rom\^ an\u a}
\subsection{Introducere}
%\u{a}
%\u{A}
%\^{a}
%\^{A}
%\^{i}
%\^{I}
%\c{s}
%\c{S}
%\c{t}
%\c{T} 


\^In aceast\u a lucrare de licen\c t\u a voi prezenta demersul realiz\u arii unei aplica\c tii web care interac\c tioneaz\u a cu un Raspberry Pi, produsul final fiind o ma\c sin\u a teleghidat\u a. \^ In cadrul lucrarii, am trecut in revist\u a tehnologiile software si hardware folosite. Aceastea se g\u asesc cu preponderen\c t\u a in capitolul 2, unde am realizat o descriere pe larg a tuturor tehnologiilor folosite. Acestea au fost alese \^ in urma experien\c teii acumulate \^ in cei 3 ani de facultate, experien\c t\u a datorat\u a materiilor dedicate acestor domenii. Am ales Raspberry Pi deoarece este un micro PC cu o putere mare de calcul \c si un consum redus de energie, dar mai ales pentru c\u a Raspberry dispune de componenta GPIO pe placa de baz\u a, care poate fi programat\u a prin intermediul  limbajelor de programare. 

Partea de server-side a aplica\c tiei este scris\u a \^ in JavaScript, folosind interpretorul node.js. Leg\u atura dintre client-side \c si server-side este realizat\u a cu web-socket-uri, deoarece datele transmise de client c\u atre server trebuie s\u a fie \^ in timp real.

Interfa\c ta web a fost realizat\u a cu HTML \c si CSS, folosind framework-ul Bootstrap.
\newpage
\subsection{Motiva\c tie}

Am ales s\u a realizez acest prototip de remote diagnostics deoarece pe viitor inten\c tionez s\u a montez un Raspberry \^ in interiorul unei ma\c sini reale. Prin intermediul acestuia vreau s\u a pot interac\c tiona cu OBD-ul ma\c sinii.


\subsection{Scop}

Scopul meu este s\u a inovez \c si s\u a \^ incerc sa \^ imbunat\u a\c tesc acest tip de aplica\c tii prin utilizarea limbajului de programare JavaScript. 
Am ales JavaScript deoarece este un tip nonconformist de implementare pentru domeniul automotive, de\c si este vast folosit \^ in alte domenii. Tocmai de aceea, am  vrut s\u a arat ca se poate folosi cu succes \c si \^ in sfera automotivului.
%-----------------------------------
%	EN
%-----------------------------------
\section{English}

\subsection{Introduction}
In this bachelor thesis, I will present the steps necessary to attain a web application that interacts with a Raspberry Pi, the final product being a controlled car. Within the paper, I defined the software and hardware technologies that I used. These are mostly found in the second chapter, where they are broadly described. They have been chosen as a result of my cumulated experience of working with them in the three years of faculty, experience that came as a result of the subjects we studied in these particular fields. I chose Raspberry Pi because it is a micro-PC with a great computing power and reduced energy consumption, but especially because Raspberry disposes of the GPIO component on the motherboard, which can be programed using programming languages. 

The server-side part of the application is written in JavaScript, using the interpreter node.js. The link between the client-side and the server-side is created with web-sockets, because the data transmitted from the client to the server has to be in real time. 

The web interface was designed with HTML and CSS, using the Bootstrap framework. 

\subsection{Motivation}

I chose to create this remote diagnosis prototype because, in the future, I intend to set up a Raspberry Pi inside a real car. Through this device, I want to be able to interact with the car’s OBD.

\subsection{Target}
My purpose is to innovate and try to improve this type of applications by using the programming language JavaScript. I chose JavaScript because it is an unconventional implementation type for the automotive field, although it is vastly used in other domains. That is why I wanted to show that it can be used successfully in the automotive area as well. 
%-----------------------------------
%	deutsch
%-----------------------------------
\section{Deutsch}

\subsection{Einleitung}
In dieser Bachelorarbeit werde ich die Herstellung einer Web-Anwendung vorstellen und die Interaktion mit einem Raspberry Pi , das Endprodukt ist eine Drohne-Maschine.In der Dokumentation habe ich geschrieben welche Software- und Hardware Technologien ich verwendet habe.Diese alle kommen in dem Kapitel 2 wo ich alle Konzepte beschrieben habe von den Web-Technologien.Sie wurden ausgewahlt nach der gewonenen Erfahrung in 3 Jahren Studium , Erfahrung fallig der Vorlesungen zu diesen Bereichen gewidmet.Ich habe mich fur Raspberry Pi entschieden weil es ein Mikro PC ist , mit einem hohen Rechenleistung und einen geringen Energieverbrauch , insbesondere weil Raspberry die GPIO-Komponente auf der Hauptplatine hat die uber die Programmiersprachen programmiert werden kann. Ein Teil der Server Seite der Anwendung ist in Javascript geschrieben nit der Verwedung von node.js . Die Verbindung zwischen Client-Seite und Server-Seite ist mit Websocket  entwickelt, weil die von dem Client an den Server gesendete Daten mussen in Realrechtzeit ankommen.

Die Web-Interface wurde mit Hilfe von HTML und CSS realisiert , mit Verwendung des Frameworks Bootsrap.


\subsection{Motivation}

Ich habe mich fur den Prototypen der Ferndiagnose entschieden weil ich in der Zukunft ein Raspberry Pi auf ein reales Auto installiert will um mit dem OBD der Maschine zu interagieren.


\subsection{Ziel}

Mein Ziel ist es inovativ zu sein , diese Art von Anwendung zu verbessern mit Hilfe der Programmiersprache JavaScript . Ich habe JavaScript gewahlt weil es eine Art der Durchfuhrung der Automibil Aussenseiter ist obwohl in anderen Gebieten ist es oft benutzt.Deshalb wollte ich zeigen das man es erfolgreich auch im Bereich der Maschinen benutzen kann.