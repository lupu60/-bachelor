% Chapter Template

\chapter{Conclusions} % Main chapter title

\label{Chapter5} % Change X to a consecutive number; for referencing this chapter elsewhere, use \ref{ChapterX}

\lhead{Chapter 5. \emph{Conclusions}} % Change X to a consecutive number; this is for the header on each page - perhaps a shortened title

%----------------------------------------------------------------------------------------
%	SECTION 1
%----------------------------------------------------------------------------------------

After concluding my paper, there are a few conclusions that I was able to draw from it. Firstly, as my practical application, as well as the thesis shows, I managed to successfully implement a Raspberry Pi on a node.js server and control a prototype car. I created this application because some day, I would like to apply it on a real car. 


Moreover, through this paper, I wanted to demonstrate the power of JavaScript and just how much it evolved in the last couple of years. Indeed, the fact that I was able to use this programming language, considered unconventional for this domain, proved that it was a viable option and it still has room to evolve in the following years. 


Another conclusion I reached was that the type of data base that I have chosen (NoSQL) was ideal for this kind of application because of its features: high speed and a small stocking spaced needed. And least, but not last, after doing my research and practically using these tools, I realised that the web technologies are no longer consecrated for the web applications, but can also be successfully used in the mobile field as well.